\section{Clustering and Routing Desgin}
\label{Section4_algorithm}

Now we are ready to describe the \emph{Contact-Prediction Clustering-based Routing} (CPCR) algorithm.
The CPCR leverages the estimated pair-wise ICT distribution (via Equation \ref{equation_label_lambda}) to predict the contact probability among nodes (using Equation \ref{equation_label_probability}), and uses such probabilities to form clusters. Based on the clustering, CPCR perform its intra- and inter-cluster routing strategies, respectively. In this section, the clustering criteria and routing process are given. Since the process is in a distribute way, we introduce the event-driven process of information-exchanging to from cluster and deliver messages in detail.
\subsection{Notation Definition}
Basic data structures and notations in CPCR is reviewed as follows.
\begin{equation}
 Node :<N_i, C_i, \{CR\}, \{GR\}, \{Msg\},\{MLog\},timer>
\end{equation}
\begin{equation}\label{xxx}
  CR :<N_i,encounters,timestamp,\sum ICT,prob>
\end{equation}
\begin{equation}\label{xxx2}
GR :<C_i,N_i,prob,ttl>
\end{equation}
\begin{equation}\label{xxx3}
Msg:<M_i,N_i,content,ttl>
\end{equation}
\begin{equation}\label{xxx4}
MLog:<M_i,\{C_i\},status,ttl>
\end{equation}
Where $N_i$ is the node ID, $C_i$ is its cluster ID, CR represents a contact record, GR donates the gateway records. Msg donates the message carried by the node. For each node, a timer is set to update the those information above periodically. In a CR, it records the encounter times by field $encounters$ and the sum of ICTs by $\sum ICT$. It also records the timestamp of latest disconnection in order to figure out next ICT.

\subsection{Clustering Criteria}
CPCR utilizes $prob_{ij}(t)$, contact probability for nodes $i$ and $j$ in time period $t$, as the clustering metric. Hereafter, $prob_{ij}(t)$ can be simplified as $p_{ij}$. The clustering criteria is given as follows:

  \textbf{Criterion 1:} We group nodes with contact probability higher than a probability threshold $\eta$ into clusters.

  \textbf{Criterion 2:} If a node conforms with the Criterion 1 for two or more clusters, the node will join the most stable cluster. To simplify, we define the cluster with larger minim of contact probability in cluster is more stable.
  
\subsection{Routing Strategy}
 
Based on dynamically formed clusters, our clustering-based routing procedure can be divided into intra-cluster and inter cluster routing procedures. If destination node is in the same cluster of a node, a single-copy routing strategy will be executed as intra-cluster routing algorithm. The local node will not deliver the packet until contacted by the destination node. Considering high contact probability inside a cluster, such direct delivery can simplify route decision making and ensure acceptable delay in the intra-cluster routing.

If the local node and the destination belong to the different clusters, inter-cluster flooding approach will be used. Every cluster is considered as an abstract "node". Therefore, it ensures that a cluster can only accept no more than one copy of a message.

\subsection{Event-driven process to form clusters and delivery messages}

Since the packets are delivered only when there exists connections between nodes, the following update processing is triggered when contacts happen.

\begin{algorithm}[!t]
\caption{Connecting}
\STATE $updateCRs()$\\
\STATE $updateGRs()$\\
\STATE $Msgs = getValidMsgs()$\\
\STATE $delivery(Msgs)$\\
\end{algorithm}

 


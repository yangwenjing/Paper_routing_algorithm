\section{Related Work}
\label{Section2_relatedwork}

%%%%%%%%%%%%%%%%%%%%%%%%%%%%%%%%%
%%��ع������޸ķ�����
%%%%%%%%%%%%%%%%%%%%%%%%%%%%%%%%%


Traditional flat routing methods, such as \emph{Direct Delivery} (DD) \cite{eKleywegtNori-6} and \emph{Epidemic Routing} \cite{gVahdatBecker-8}, become not scalable to large-scale DTNs. Meanwhile, clustering-based approaches have long been considered as an effective approach to reduce network overhead and improve scalability in traditional mobile ad hoc networks \cite{hAgarwalMotwani-9,iLiuLiu-10,jWangMi-11,kWhitbeckConan-12} with relatively stable topology and more communication opportunity.

Clustering in DTNs is unique \cite{uHaHongyi-22,kernal2005survey}, because the network topology is not always fully connected. As a result, it becomes much more challenging to formulate clusters and ensure their stability. When we consider an urban DTN, the nodal scale and mobility will further increase the cost of maintaining clusters.


In summary, the key challenge of hierarchical routing is the clustering method, especially in large scale DTNs. To adapt to the scenario of large scale urban DTNs and solve the routing problem, a cluster-based routing algorithm is proposed.
